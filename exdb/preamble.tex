\usepackage{ifxetex}
\usepackage{ifluatex}
\usepackage{ifthen}
\usepackage{babel}
\usepackage{amsfonts,amsmath,amssymb}

\ifxetex
  \RequirePackage{xltxtra}
  \RequirePackage{fontspec}
  \defaultfontfeatures{Mapping=tex-text} % damit -- und --- funktionieren
  \RequirePackage{unicode-math}
  \setromanfont[Numbers=Proportional]{Linux Libertine O}
  \setsansfont[Numbers=Proportional]{Linux Biolinum O} %
  \setmonofont{Inconsolata}
  \setmathfont[StylisticSet=0]{XITS Math}
\else
  \ifluatex
    \RequirePackage{fontspec}
    \defaultfontfeatures{Mapping=tex-text} % damit -- und --- funktionieren
    \setromanfont[Numbers=Proportional]{Linux Libertine O}
    \setsansfont[Numbers=Proportional]{Linux Biolinum O} %
    \RequirePackage{unicode-math}
    \setmathfont[StylisticSet=0]{XITS Math}
  \else
    \RequirePackage[utf8]{inputenc}
    \usepackage{csquotes}
    \RequirePackage[T1]{fontenc}
    \RequirePackage{libertine}
  \fi
\fi

\usepackage{enumitem}
\setlist[enumerate,1]{label=\alph*)}
\setlist[enumerate,2]{label=\arabic*.}

\usepackage{tikz}

% some handy styles for drawing graphs
\tikzstyle{v}=[draw,circle,inner sep=.8mm,minimum size=4mm] % vertex
\tikzstyle{a}=[->,sloped] % arc
\tikzstyle{e}=[sloped] % edge

\usepackage{mathtools}
\usepackage{IEEEtrantools}
\usepackage{xspace}
\usepackage[amssymb]{SIunits}


% Mathematical symbols and abbreviations

% number sets
\newcommand{\Z}{\mathbb{Z}}
\newcommand{\Q}{\mathbb{Q}}
\newcommand{\N}{\mathbb{N}}
\newcommand{\R}{\mathbb{R}}
\ifx\C\undefined
  \newcommand{\C}{\mathbb{C}}
\else
  \renewcommand{\C}{\mathbb{C}}
\fi
\newcommand{\F}{\mathbb{F}}

\renewcommand{\O}{\mathcal{O}} % Landau symbol
\newcommand{\E}{\mathbb{E}} % expectation


\newcommand{\abs}[1]{\left\lvert #1 \right\rvert} % absolute value
\newcommand{\norm}[1]{\left\| #1 \right\|} % norm
\newcommand{\gauss}[1]{\left\lfloor #1 \right\rfloor} % floor operator
\newcommand{\op}[1]{\operatorname{#1}} % just an abbreviation

% typographically correct punctuation marks in math mode
\newcommand\tp{\text{.}}
\newcommand\tk{\text{,}}
\newcommand\ts{\text{;}}

\newcommand\definedAs{\vcentcolon=} % := with vertically centered column (requires mathtools)

% some operators
\DeclareMathOperator{\ggT}{ggT} % größter gemeinsamer Teiler
\DeclareMathOperator{\kgV}{kgV} % kleinstes gemeinsames Vielfaches
\DeclareMathOperator*{\argmin}{arg\,min}
\DeclareMathOperator*{\argmax}{arg\,max}
\DeclareMathOperator{\conv}{conv} % convex hull
\DeclareMathOperator{\aff}{aff} % affine hull
\DeclareMathOperator{\eq}{eq} %equality set (polyhedra)
\DeclareMathOperator{\rank}{rank}
\DeclareMathOperator{\spn}{span}

% common abbreviations

\newcommand{\zB}{\text{z.\,B.}\xspace}
\newcommand\st{\text{s.\,t.}\xspace}
\newcommand\ie{\text{i.\,e.}\xspace}
\newcommand\eg{\text{e.\,g.}\xspace}

% references (package prettyref)
\usepackage{prettyref}
\newrefformat{eq}{\textup{(\ref{#1})}}
\newrefformat{def}{Definition~\ref{#1}}
\newrefformat{lemma}{Lemma~\ref{#1}}
\iflanguage{ngerman}{
 	\newrefformat{alg}{Algorithmus~\ref{#1}}
 	\newrefformat{fig}{Abbildung~\ref{#1}}
 	\newrefformat{rem}{Bemerkung~\ref{#1}}
 	\newrefformat{example}{Beispiel~\ref{#1}}
 	\newrefformat{chapter}{Kapitel~\ref{#1}}
 	\newrefformat{sec}{Abschnitt~\ref{#1}}
 	\newrefformat{thm}{Satz~\ref{#1}}
 	\newrefformat{obs}{Beobachtung~\ref{#1}}
 	\newrefformat{cor}{Korollar~\ref{#1}}
 	\newrefformat{line}{Zeile~\ref{#1}}
 	\newrefformat{table}{Tabelle~\ref{#1}}
}{
	\newrefformat{alg}{Algorithm~\ref{#1}}
	\newrefformat{fig}{Figure~\ref{#1}}
	\newrefformat{rem}{Remark~\ref{#1}}
	\newrefformat{example}{Example~\ref{#1}}
	\newrefformat{chapter}{Chapter~\ref{#1}}
	\newrefformat{sec}{Section~\ref{#1}}
	\newrefformat{thm}{Theorem~\ref{#1}}
	\newrefformat{obs}{Observation~\ref{#1}}
	\newrefformat{cor}{Corollary~\ref{#1}}
	\newrefformat{line}{Line~\ref{#1}}
	\newrefformat{table}{Table~\ref{#1}}
}


% theorem environments
\usepackage[amsmath,thmmarks]{ntheorem}
\qedsymbol{\ensuremath{_\Box}}

% statements in italics
\theorembodyfont{\itshape}
\theoremseparator{:}

\theoremstyle{plain} % name followed by number

\newtheorem{theorem}{Theorem}
\newtheorem{satz}{Satz}
\newtheorem{lemma}[theorem]{Lemma}
\newtheorem{corollary}[theorem]{Corollary}
\newtheorem{algo}[theorem]{Algorithm}
\newtheorem{observation}[theorem]{Observation}

% examples, definitions and so on in normal font and with a nice triangle at the end
\theorembodyfont{\normalfont}
\theoremsymbol{\ensuremath{_\lhd}}
\newtheorem{definition}[theorem]{Definition}
\newtheorem{example}[theorem]{Example}
\newtheorem{beispiel}[theorem]{Beispiel}
\newtheorem{problem}[theorem]{Problem}
\newtheorem{remark}[theorem]{Remark}

\theoremsymbol{}
\theoremstyle{nonumberplain}
\theoremseparator{.}
\theoremheaderfont{\normalfont\bfseries\itshape}
\theoremsymbol{\ensuremath{_\Box}}
\newtheorem{proof}{Proof}
\newtheorem{beweis}{Beweis}
\theoremseparator{}
\theoremindent0cm
\newtheorem{fakeproof}{}